\section{Introducción}

Luego de instalar una webcam, siempre es bueno probar su funcionamiento. Para tal fin los diseñadores de Apple incluyen PhotoBooth en sus instalaciones de OSX. PhotoBooth permite aplicar una gran variedad de filtros y junto con su calidad de imagen hacen de este software una aplicaci\'on de entretenimiento

Generalmente, los usuarios de Microsoft cuentan con una opci\'on clara a la hora de testear su webcam, dado que cada webcam viene con sus drivers y su software propietario para administrarla.

Sin embargo, los usuarios de sistemas operativos basados en Linux y BSD no cuentan con estas ventajas. Son pocas las empresas que diseñan webcams e incluyen drivers para Linux, aunque existen drivers gen\'ericos (por ejemplo qc-usb y libv4l). A la hora de interactuar con la webcam, los usuarios Linux recurren a la comunidad en busca de un software acorde a sus necesidades. Si bien existen soluciones implementadas (Camorama, Cheese, WxCam, entre otras) a menudo no vienen inclu\'idas por defecto en las distribuciones mas populares de Linux y aun as\'i, los efectos de video disponibles son limitados y lentos.

La finalidad de este trabajo es desarrollar un conjunto de filtros para im\'agenes, aplicables al video de captura de una webcam. Para tal fin, consideramos las alternativas Open Source disponibles en la comunidad. De las aplicaciones disponibles, decidimos encarar este proyecto utilizando WxCam por su simplicidad y versatilidad.
